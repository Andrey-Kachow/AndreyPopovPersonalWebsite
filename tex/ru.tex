
%-------------------------
% Resume in Latex
% Author: Andrey Popov
% Based off of: https://github.com/jakegut/resume
% License: MIT
%------------------------

\documentclass[letterpaper,11pt]{article}

\usepackage{latexsym}
\usepackage[empty]{fullpage}
\usepackage{titlesec}
\usepackage{marvosym}
\usepackage[usenames,dvipsnames]{color}
\usepackage{verbatim}
\usepackage{enumitem}
\usepackage[hidelinks]{hyperref}
\usepackage{fancyhdr}
\usepackage[english, russian]{babel}
\usepackage{tabularx}
\usepackage{ragged2e}

\input{glyphtounicode}

\pagestyle{fancy}
\fancyhf{} % clear all header and footer fields
\fancyfoot{}
\renewcommand{\headrulewidth}{0pt}
\renewcommand{\footrulewidth}{0pt}

% Adjust margins
\addtolength{\oddsidemargin}{-0.5in}
\addtolength{\evensidemargin}{-0.5in}
\addtolength{\textwidth}{1in}
\addtolength{\topmargin}{-.5in}
\addtolength{\textheight}{1.0in}

\urlstyle{same}

\raggedbottom
\raggedright
\setlength{\tabcolsep}{0in}

% Sections formatting
\titleformat{\section}{
  \vspace{-4pt}\scshape\raggedright\large
}{}{0em}{}[\color{black}\titlerule \vspace{-5pt}]

% Ensure that generate pdf is machine readable/ATS parsable
\pdfgentounicode=1

%-------------------------
% Custom commands
\newcommand{\resumeItem}[1]{
  \item\small{
    {#1 \vspace{-2pt}}
  }
}

\newcommand{\plusplus}{$\mathbf{++}$}

\newcommand{\resumeSubheading}[4]{
  \vspace{-2pt}\item
    \begin{tabular*}{0.97\textwidth}[t]{l@{\extracolsep{\fill}}r}
      \textbf{#1} & #2 \\
      \textit{\small#3} & \textit{\small #4} \\
    \end{tabular*}\vspace{-7pt}
}

\newcommand{\resumeSubSubheading}[2]{
    \item
    \begin{tabular*}{0.97\textwidth}{l@{\extracolsep{\fill}}r}
      \textit{\small#1} & \textit{\small #2} \\
    \end{tabular*}\vspace{-7pt}
}

\newcommand{\resumeProjectHeading}[2]{
    \item
    \begin{tabular*}{0.97\textwidth}{l@{\extracolsep{\fill}}r}
      \small#1 & #2 \\
    \end{tabular*}\vspace{-7pt}
}

\newcommand{\resumeSubItem}[1]{\resumeItem{#1}\vspace{-4pt}}

\renewcommand\labelitemii{$\vcenter{\hbox{\tiny$\bullet$}}$}

\newcommand{\resumeSubHeadingListStart}{\begin{itemize}[leftmargin=0.15in, label={}]}
\newcommand{\resumeSubHeadingListEnd}{\end{itemize}}
\newcommand{\resumeItemListStart}{\begin{itemize}}
\newcommand{\resumeItemListEnd}{\end{itemize}\vspace{-5pt}}

%-------------------------------------------
%%%%%%  RESUME STARTS HERE  %%%%%%%%%%%%%%%%%%%%%%%%%%%%

\begin{document}

\begin{center}
    \textbf{\huge \scshape Попов Андрей Константинович} \\ \vspace{1pt}
    \href{mailto:x@x.com}{\underline{andr10xp@gmail.com}} 
    $|$ 
    \href{https://github.com/Andrey-Kachow}{\underline{Github: Andrey-Kachow}} $|$
    \href{http://andreypopov.xyz/}{\underline{andreypopov.xyz}}
    $|$
    \href{https://www.linkedin.com/in/andrey-popov-10x/}{\underline{Linkedin: andrey-popov-10x}} 
\end{center}

\begin{justify}
Целеустремленный ИТ-специалист с глубокими познаниями алгоритмов и структур данных и широким набором языков, включая Python, Java, JavaScript (React).
Я разбираюсь в Web-разработке, Бэкенде, облачных сервисах, и CI/CD \& Agile.
Благодаря своему трудолюбию и желанию учиться новому, я могу решать сложнейшие и незнакомые задачи ответственно, креативно и с вниманием к деталяим.
\end{justify}

%-----------EXPERIENCE-----------
\section{Опыт Работы}
  \resumeSubHeadingListStart

    \resumeSubheading
      {Програмный Инженер (6 мес.)}{Апрель 2023 -- Сентябрь 2023}
      {Компания ``Minimax Labs Limited''}{London, UK}
      \resumeItemListStart
        \resumeItem{Добился успешной демонстрации функционала клиенту, разработав код быстно и обеспечив ранний релиз}
        \resumeItem{Разрабатывал на Java приложение для газового бизнеса, Основанное на Eclipse Platform}
        \resumeItem{Создал новый вебсайт компании, трепетно и внимательно относясь к меняющимся бизнес-требованиям}
        \resumeItem{Реализовал множество новых UI компонентов для программы оптимизации транспортировки газа}
        \resumeItem{Ответственно подошел и справился с задачей презентации своего проекта клиенту в видеозвонке}
        \resumeItem{Обеспечил стабильную работу системы, дополняя код грамотными JUnit тестами}
      \resumeItemListEnd

    \resumeSubheading
      {Куратор студентов и Ассистент Преподавателя (6 мес.)}{Октябрь 2022 -- Март 2023}
      {Кафедра компьютерных наук ``Imperial College London''}{London, UK}
      \resumeItemListStart
        \resumeItem{Организовывал и проводил воркшопы по Git, SQL и другим инструментам для 17 студентов}
        \resumeItem{Усилил навык коммуникабельности, преподавая вживую Java, Kotlin и Haskell первокурсникам}
        \resumeItem{Проводил Code Review, проверяя курсовые и предоставлял индивидуальную поддержку и менторство}
    \resumeItemListEnd

    \resumeSubheading
    {Частный Преподаватель Программирования (2 года)}{Июль 2022 -- Настоящее}
    {Самозанятый}{Дистанционно}
    \resumeItemListStart
        \resumeItem{Развил умение находить подход к людям и объяснять сложные вещи просто и понятно более 20 студентам}
        \resumeItem{Обучая учеников Python, Java, C/C\plusplus, помог людям и собрал много положительных отзывов}
    \resumeItemListEnd

  \resumeSubHeadingListEnd

%-----------PROJECTS-----------
\section{Проекты}
    \resumeSubHeadingListStart

    \resumeProjectHeading
      {\textbf{Pokemon Help Centre} $|$ \emph{Python, Django, Flutter, Sqlite}}{Июнь 2024 -- Июль 2024}
      \resumeItemListStart
        \resumeItem{Разработал приложение Support-чата с сотрудниками-покемонами, загружая информацию с PokeAPI}
        \resumeItem{Написал бэкенд на Django, включая REST framework и базу данных SQLite}
        \resumeItem{Реализовал мобильное приложение на Flutter, с личным кабинетом и чатом поддержки}
      \resumeItemListEnd
    
      \resumeProjectHeading
          {\textbf{Scaling Extreme Startup} $|$ \emph{Python, Flask, React.js, Docker, AWS, MongoDB}}{Октябрь 2022 -- Январь 2023}
          \resumeItemListStart
            \resumeItem{Working in a group of six created an educational tool for an MSc course at Imperial College}
            \resumeItem{Командой из 6 человек разработали образовательный инструмент взятый Imperial College в обращение}
            \resumeItem{Реализовал бэкенд на Flask REST API, и фронтендом на React объединенное в Docker контейнер}
            \resumeItem{Используя AWS Lambda, SQS и DynamoDB сделал систему горизонтально масштабируемой}
          \resumeItemListEnd

    \resumeProjectHeading
        {\textbf{Проект MmmmBoxes} $|$ \emph{Python, Flask, JavaScript, Pytest, SQL (Postgres)}}{Май 2022 -- Июнь 2022}
        \resumeItemListStart
            \resumeItem{Разработал Веб-ПО для обеспечения хранения и мониторинга посылок в общежития ВУЗов}
            \resumeItem{Насроил CI/CD пайплайн, тестируюший и Flask Веб-приложение на платформу Heroku}
            \resumeItem{Работал в команде и собирал обратную связь у пользователей и клиентов}
        \resumeItemListEnd
          
  \resumeProjectHeading
      {\textbf{SAT: Платформа Терапии} $|$ \emph{Python, Flask, JavaScript, Unity, SQLite}}{Октябрь 2023 -- Июнь 2024}
      \resumeItemListStart
        \resumeItem{Разработал кросс-платформ игру на Unity, помогающую пройти психотерапевтический протокол SAT}
        \resumeItem{Написал Бэкенд на Python Flask и обеспечил непрерывную работу проекта и регулярные релизы}
      \resumeItemListEnd
          
    \resumeSubHeadingListEnd

%-----------EDUCATION-----------
\section{Образование}
  \resumeSubHeadingListStart
    \resumeSubheading
      {Imperial College London}{London, UK}
      {Магистр Компьютерных Наук (MEng in Computing)}{2020 -- 2024}
  \resumeSubHeadingListEnd
  
%-----------PROGRAMMING SKILLS-----------
\section{Технические Навыки}
 \begin{itemize}[leftmargin=0.15in, label={}]
    \small{\item{
     \textbf{Языки}{: Python, SQL, JavaScript, Java, C\#, C/C++, Kotlin, HTML/CSS, Haskell} \\
     \textbf{Фреймворки}{: Django, React, Pytest, Eclipse EMF, JUnit, Flask, Flutter, Android SDK} \\
     \textbf{Инструменты}{: Git, Docker, AWS, GitLab-CI, Firebase, Unity, Godot, Apache HTTP Server} \\
     \textbf{Библиотеки}{: Requests, Boto3, NumPy, Pandas, Psycopg2, SDL2, SWT, Retrofit, TKinter, ANTLR} \\
    }}
 \end{itemize}
%-------------------------------------------
\end{document}
